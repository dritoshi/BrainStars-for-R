%\VignetteIndexEntry{brainstars}
%\VignetteDepends{brainstars}
%\VignetteKeywords{brainstars}
%\VignettePackage{brainstars}
\documentclass[12pt,fullpage]{article}

\usepackage{amsmath,epsfig,pstricks,fullpage}
\usepackage{hyperref}
\usepackage{url}
\usepackage[authoryear,round]{natbib}

\newcommand{\Rfunction}[1]{{\texttt{#1}}}
\newcommand{\Robject}[1]{{\texttt{#1}}}
\newcommand{\Rpackage}[1]{{\textit{#1}}}
\newcommand{\Rclass}[1]{{\textit{#1}}}
\newcommand{\Rmethod}[1]{{\textit{#1}}}

\author{Itoshi NIKAIDO$^\ddagger$\footnote{dritoshi@gmail.com}}
\usepackage{Sweave}
\begin{document}
\title{Using the BrainStars package}
\maketitle
\begin{center}$^\ddagger$The RIKEN Center for Developmental Biology
\end{center}


\tableofcontents
%%%%%%%%%%%%%%%%%%%%%%%%%%%%%%%%%%%%%%%%%%%%%%
\section{Overview of BrainStars} 

BrainStars is a quantitative expression database of the adult
mouse brain. The database has genome-wide expression profile at 51
adult mouse CNS regions.

For 51 CNS regions, slices (0.5-mm thick) of mouse brain were cut on a
Mouse Brain Matrix, frozen, and the specific regions were punched out
bilaterally with a microdissecting needle (gauge 0.5 mm) under a
stereomicroscope. For each region, we took samples every 4 hours,
starting at ZT0 (Zeitgaber time 0; the time of lights on), for 24
hours (6 time-point samples for each region), and we pooled the
samples from the different time points. We independently sampled each
region twice (n=2).

These samples were purified their RNA, and measured with Affymetrix
GeneChip Mouse Genome 430 2.0 arrays. Expression values were then
summarized with the RMA method. After several analysis with the
expression data, the data and analysis results were stored in the
BrainStars database.

BrainStars database has a REST API to query gene expression data and
some kind of figures written by Dr. Takeya Kasukawa. This package is
wrapper for BrainStars REST API in R. BrainStars data, images and
texts (excluding ABA data and images) are licensed under a Creative
Commons Attribution 2.1 Japan License.

\subsection{Gene Expression}
A gene expression describes the expression level of gene on the array. This entry returns \Rclass{ExpressionSet} object.

\subsection{Figure}
BrainStars can be export five type of figures following:
\begin{itemize}
  \item exprgraph Barplot of gene expression level by brain regions
  \item exprmap Gene expression level on brain slice images by brain regions
  \item switchgraph ...
  \item switchhist ...
\end{itemize}

\subsection{Marker}
Marker genes were ones whose levels in a specific CNS region are higher (or lower) than in others. These marker genes were identified based on the multi-state gene analysis.

\subsection{Search}
 You can search for entries with gene name, gene symbol, synonym, Entrez GeneID, and probe set ID. (2) Push "search" button for performing keyword search.

\section{Getting Started using brainstars}

Getting expression data from BrainStars is easy.  \Rfunction{getBrainStarsExpression} interprets its input to determine how to get the data from BrainStars and then parse the data into \Rclass{ExpressionSet} object.  Usage is quite simple:

\begin{Schunk}
\begin{Sinput}
> library(brainstars)
\end{Sinput}
\end{Schunk}

This loads the brainstars library.

\begin{Schunk}
\begin{Sinput}
> my.eset <- getBrainStarsExpression("1439627_at")
\end{Sinput}
\end{Schunk}

Now, \Robject{my.eset} contains the R data structure (of class \Rclass{ExpressionSet}) that represents the entry 1439627\_at from BrainStars.

We can retrive some kind of barplots from BrainStars.

\begin{Schunk}
\begin{Sinput}
> getBrainStarsFigure("1439627_at", "exprgraph")
\end{Sinput}
\begin{Soutput}
Downloading...
Done.
\end{Soutput}
\begin{Sinput}
> getBrainStarsFigure("1439627_at", "exprmap")
\end{Sinput}
\begin{Soutput}
Downloading...
Done.
\end{Soutput}
\begin{Sinput}
> getBrainStarsFigure("1439627_at", "switchgraph")
\end{Sinput}
\begin{Soutput}
Downloading...
Done.
\end{Soutput}
\begin{Sinput}
> getBrainStarsFigure("1439627_at", "switchhist")
\end{Sinput}
\begin{Soutput}
Downloading...
Done.
\end{Soutput}
\end{Schunk}

You can find "id.type.png" files in current directory.

\begin{Schunk}
\begin{Sinput}
> recep.list <- getBrainStarsSearch("receptor/10,5")
\end{Sinput}
\begin{Soutput}
Downloading...
Done.
\end{Soutput}
\begin{Sinput}
> recep.count <- getBrainStarsSearch("receptor/count")
\end{Sinput}
\begin{Soutput}
Downloading...
Done.
\end{Soutput}
\end{Schunk}

Markers,
\begin{Schunk}
\begin{Sinput}
> my.genes.json <- getBrainStarsMarker("high/LS/count")
\end{Sinput}
\begin{Soutput}
Downloading...
http://brainstars.org/marker/high/LS/count?content-type=application/json 
Done.
\end{Soutput}
\end{Schunk}

Keyword search,
\begin{Schunk}
\begin{Sinput}
> recep.list <- getBrainStarsSearch("receptor/10,5")
\end{Sinput}
\begin{Soutput}
Downloading...
Done.
\end{Soutput}
\begin{Sinput}
> recep.count <- getBrainStarsSearch("receptor/count")
\end{Sinput}
\begin{Soutput}
Downloading...
Done.
\end{Soutput}
\end{Schunk}

\section{Brainstars Data Structures}
The brainstars data structures ...

\section{Conclusion}
The brainstars package provides a ...
\end{document}
